\documentclass[presentation,aspectratio=169]{beamer}

\usepackage[utf8]{inputenc}
\usepackage[T1]{fontenc}
\usepackage{tikz}
\usetheme[showmaxslides, darkmode]{pureminimalistic}
\usepackage{../src/equalspeak}
\usepackage{listings}

\title{Equalspeak: math-aware \LaTeXe ~accessibility (version 0.1 alpha)}
\author{James B. Wilson}
\date{\today}
\setbeamertemplate{footline}{}
\begin{document}


% \begin{frame}
%     %% Math text
%     \begin{align*}
%         \pmat{ 1 \\ 1 }
%     \end{align*}    
    
%     \end{frame}
    
\begin{frame}
    \titlepage
\end{frame}
\begin{frame}
    The following is a demonstration of using \LaTeXe ~to 
    help make math communication accessible.
\end{frame}

\begin{frame}[fragile]
   In the first example I create the following command in \LaTeXe.
   Don't worry about what it means but know that many math content 
   creators communicate in this way.\\
   \rule{\textwidth}{1pt}
\begin{lstlisting}
\begin{align*}
    \pmat{ 5.2 \\ 0 \\ 4.3 }+\pmat{ 1 \\ 1 \\ 1}-
    a\cdot \pmat{ 5.2 \\ 0 \\ 4.3}+b\pmat{ 1 \\ 1 \\ 1 }
\end{align*}    
\end{lstlisting}    
            
\end{frame}

\begin{frame}
%% Math text
Traditional \LaTeXe converts that code into visible symbols 
as shown below, but when read aloud by standard software 
it becomes confused, misleading, even wrong.\\
\rule{\textwidth}{1pt}
\begin{align*}
    span\left\{\pmat{ 5.2 \\ 0 \\ 4.3 }, \pmat{ 1 \\ 1 \\ 1}-
    a\cdot \pmat{ 5.2 \\ 0 \\ 4.3}\right\}
\end{align*}    
\end{frame}

\begin{frame} 
    Now you may not understand the math, and that is not the point!\\[20pt]

    But I can tell you that the way I read the symbols out loud was 
    wrong, very wrong.  It it can even be worse on other read-aloud 
    programs. And that is because math symbols are not read left-to-right, 
    top-to-bottom but use a mix of directions, orders of operations, and 
    historical pronunciations.

\end{frame}

\begin{frame}
%% Alt text
Now without changing any of the content, we instead use Equalspeak 
to tell \LaTeXe ~ to 
instead print out words that would be associated with reading the 
mathematics aloud.  Our example comes out like this.\\[20pt]
\rule{\textwidth}{1pt}

\speakifytext
\begin{center}
    span\left\{\pmat{ 5.2 \\ 0 \\ 4.3 }, \pmat{ 1 \\ 1 \\ 1}-
    a\cdot \pmat{ 5.2 \\ 0 \\ 4.3}\right\}
\end{center}
\endspeakifytext
\end{frame}

\begin{frame}
    Yes, that last example might still be confusing, but now 
    all that confusion is down to what we do or don't yet know about 
    the math.  It is in fact a faithful description of what the 
    visual equations tell a reader, but done in words.\\[20pt]
    
    
    The good 
    new is that math content creators really know how to explain the math 
    confusion, so we can let them now do what their good at.
\end{frame}

\begin{frame}
    It is important to explain that the original content, the 
    math expression, is the same snippet of \LaTeXe ~ code in both examples.\\[20pt]
    
    Thus content creators can write their usual formulas 
    and then at a later time decide how to output the results, 
    producing perhaps one or more versions appropriate to the needs 
    of accessibility.\\
    
    There can even be hybrid versions.    
\end{frame}

\begin{frame}[fragile]
    The next example concerns images, which because those are 
    created outside of \LaTeXe ~will require content creators to help 
    label the image with useful alternative text.  Here is our example.\\[20pt]
    
    \rule{\textwidth}{1pt}
\begin{lstlisting}
\includegraphic[
    width=2in, 
    alt={A cube with exposed faces shown as a gradient of color 
    changing from red in the lower left, to yellow lower 
    middle, green lower right, cyan upper right, 
    blue upper middle, purple upper left, and 
    white in the center.}]
    {colorcube.jpg}
\end{lstlisting}
        
\end{frame}

\begin{frame}
    The result in a visual only mode is an image.  In some viewers 
    like Adobe Acrobat and Apple Preview this will have also assigned alternate text to the image 
    so that on mouse over or clicking we get the clarifying information.\\[20pt]

    \rule{\textwidth}{1pt}
    \wimgtxt[width=2in]{colorcube.jpg}{
        A cube with exposed faces shown as a gradient of color 
        changing from red in the lower left, to yellow lower 
        middle, green lower right, cyan upper right, 
        blue upper middle, purple upper left, and 
        white in the center.
    }
\end{frame}

\begin{frame}
    By changing the mode to be read-aloud, \LaTeXe ~can instead 
    remove the image and print just the alternate text. \\[20pt]

    \rule{\textwidth}{1pt}
    % \wimgtxt[width=2in]{colorcube.jpg}{
    %     A cube with exposed faces shown as a gradient of color 
    %     changing from red in the lower left, to yellow lower 
    %     middle, green lower right, cyan upper right, 
    %     blue upper middle, purple upper left, and 
    %     white in the center.
    % }
    A cube with exposed faces shown as a gradient of color 
        changing from red in the lower left, to yellow lower 
        middle, green lower right, cyan upper right, 
        blue upper middle, purple upper left, and 
        white in the center.
\end{frame}

\begin{frame}
    The system is not perfect, in fact there are still a lot of hacks.
    But we know enough to know it works and will get better.\\[20pt]

    Content creators of mathematics 
    can use what we have with minimal changes to their normal and 
    practiced modes of working and achieve this result because 
    \LaTeXe ~content is dripping with the information necessary 
    to communicate in numerous modalities. \\[20pt]

    If you have questions or suggestions 
    we are open to reading \textbf{and} listening.
\end{frame}


\end{document}
