%% The following is inspired by 
%% https://tex.stackexchange.com/questions/454944/is-there-screen-reader-software-or-a-built-in-method-that-supports-latex-equat/505462#505462
%% I'd like to make it work on a broader range of math formulas such as matrices, roots, norms, 
%% As a basic target I'd like to be able to use this to speak out the equations of linear algebra

\documentclass{article}

\usepackage[T1]{fontenc}
\usepackage[margin=1in]{geometry}
\usepackage{amsmath}\usepackage{amssymb, amsthm}

\newcommand{\mymat}[1]{\begin{pmatrix} #1 \end{pmatrix}}


\begin{document}

\section{Making \LaTeX{} Math Audibly Legible}

We are trying something new to make \LaTeX\ more readable. We have some inline math, for example, we know that \(3^2 + 4^2 = 5^{2}\). Also \(x^2 + xy = z\) and the following:

Another \(inline\) example. Here is a simple equation:
\[
      x^2 + y^2 = z^2 + 1 
\]

Now I'm letting the copilot fly:

Here is a summary of the Riemann Hypothesis: 
\[
    \zeta(s) = \sum_{n=1}^{\infty} \frac{1}{n^s} = \prod_{p \text{ prime}} \frac{1}{1 - p^{-s}}
\]
The Riemann Hypothesis states that all non-trivial zeros of the zeta function
lie on the line \(\Re(s) = \frac{1}{2}\). It is one of the most famous unsolved
problems in mathematics. However, it is not the only unsolved problem in
mathematics. There are many others, such as the Goldbach Conjecture, the Collatz
Conjecture, and the Twin Prime Conjecture. Most importantly, the Riemann
Hypothesis is a conjecture, not a theorem. It has not been proven, but it has
not been disproven either. It is a very difficult problem, and it has been open
for over \(15\)0 years. One day, I hope that I will be able to solve it.

\begin{equation}
       x_i^2 + y^2 = (z^2 + 1)
    \end{equation}

        \[\sqrt{2}
         \quad \text{and} \quad   \]


    \begin{equation*}
        \mymat{ 1 & 2 \\ 3 & 4 }
     \end{equation*} 
\begin{equation}
    x_i^2 + y^2 = (z^2 + 1)
 \end{equation}
Or this one:
    \[
       Z = \frac{x+1}{2}
    \]
\end{document}
